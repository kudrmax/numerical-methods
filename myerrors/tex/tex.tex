\documentclass[a4paper,11pt]{article}

% \usepackage[smartEllipses]{markdown}
% \usepackage[hashEnumerators,smartEllipses]{markdown}
% \usepackage[hashEnumerators,smartEllipses,hybrid]{markdown}
\usepackage[hybrid]{markdown}
\usepackage[export]{adjustbox} % рамки вокруг фото
\usepackage{minted} % Для вставки кода
% \usepackage{xpatch}
% \xpretocmd{\inputminted}{\par\vspace{1em}}{}{}
% \xapptocmd{\inputminted}{\par\vspace{1em}}{}{}

%%% Мои команды
\newcommand{\insertMyCode}[2]{ % вставка кода
    % #1 — язык
    % #2 — путь до файла
    \inputminted[
    % framesep=2mm,
    baselinestretch=1.2,
    bgcolor=LightGray!30,
    fontsize=\footnotesize,
    linenos
    ]{#1}{#2}
}


%%% Работа с русским языком
\usepackage{cmap} % поиск в PDF
\usepackage{mathtext} % русские буквы в фомулах
\usepackage[T2A]{fontenc} % кодировка
\usepackage[utf8]{inputenc} % кодировка исходного текста
\usepackage[english,russian]{babel} % локализация и переносы

%%% Дополнительная работа с математикой
\usepackage{amsmath,amsfonts,amssymb,amsthm,mathtools} % AMS
\usepackage{icomma} % "Умная" запятая: $0,2$ —- число, $0, 2$ —- перечисление

%%% Перенос знаков в формулах (по Львовскому)
\newcommand*{\hm}[1]{#1\nobreak\discretionary{}
{\hbox{$\mathsurround=0pt #1$}}{}}

%%% Работа с картинками
\usepackage{graphicx} % Для вставки рисунков
\graphicspath{{images/}} % папки с картинками
% \setlength\fboxsep{3pt} % Отступ рамки \fbox{} от рисунка
% \setlength\fboxrule{1pt} % Толщина линий рамки \fbox{}
% \usepackage{wrapfig} % Обтекание рисунков текстом

%%% Работа с таблицами
\usepackage{array,tabularx,tabulary,booktabs} % Дополнительная работа с таблицами
\usepackage{longtable} % Длинные таблицы
\usepackage{multirow} % Слияние строк в таблице

%%% Теоремы
% \theoremstyle{plain} % Это стиль по умолчанию, его можно не переопределять.
\newtheorem{theorem}{Теорема}[section]
\newtheorem{proposition}[theorem]{Утверждение}

\theoremstyle{definition} % "Определение"
\newtheorem{corollary}{Следствие}[theorem]
\newtheorem{problem}{Задача}[section]

\theoremstyle{remark} % "Примечание"
\newtheorem*{nonum}{Решение}

%%% Программирование
\usepackage{etoolbox} % логические операторы

%%% Страница
\usepackage{extsizes} % Возможность сделать 14-й шрифт
\usepackage{geometry} % Простой способ задавать поля
\geometry{top=25mm}
\geometry{bottom=35mm}
\geometry{left=20mm}
\geometry{right=20mm}

\usepackage{fancyhdr} % Колонтитулы
\pagestyle{fancy}
\renewcommand{\headrulewidth}{0mm} % Толщина линейки, отчеркивающей верхний колонтитул
% \lfoot{Нижний левый}
% \rfoot{Нижний правый}
% \rhead{Верхний правый}
% \chead{Верхний в центре}
% \lhead{Верхний левый}
\cfoot % По умолчанию здесь номер страницы

\usepackage{lastpage} % Узнать, сколько всего страниц в документе.

\usepackage{soul} % Модификаторы начертания

\usepackage{setspace} % Интерлиньяж
%\onehalfspacing % Интерлиньяж 1.5
%\doublespacing % Интерлиньяж 2
%\singlespacing % Интерлиньяж 1

\usepackage{hyperref}
\usepackage[usenames,dvipsnames,svgnames,table,rgb]{xcolor}
\hypersetup{ % Гиперссылки
unicode=true, % русские буквы в раздела PDF
pdftitle={Заголовок}, % Заголовок
pdfauthor={Автор}, % Автор
pdfsubject={Тема}, % Тема
pdfcreator={Создатель}, % Создатель
pdfproducer={Производитель}, % Производитель
pdfkeywords={keyword1} {key2} {key3}, % Ключевые слова
colorlinks=true, % false: ссылки в рамках; true: цветные ссылки
linkcolor=red, % внутренние ссылки
citecolor=green, % на библиографию
filecolor=magenta, % на файлы
urlcolor=cyan % на URL
}

\usepackage{multicol} % Несколько колонок

%%% Хз что это (потом разобраться)
% \usepackage{upgreek}
\usepackage{cite}
\usepackage{csquotes} % Еще инструменты для ссылок
%Смотри источник 1 \cite{qwerty,Fama,Fama2}.
%\renewcommand{\refname}{Список источников}  % По умолчанию %"Список литературы" (article)
%\renewcommand{\bibname}{Литература}  % По умолчанию "Литература" (book и report)
%\renewcommand{\familydefault}{\sfdefault} % Начертание шрифта

\begin{document}


\begin{titlepage} % начало титульной страницы
\pagestyle{empty}
\begin{center}

\Large
\textbf{Федеральное государственное автономное образовательное учреждение высшего образования\\
«Национальный исследовательский университет\\
«Высшая школа экономики»}\\
\vspace{5mm}

\Large
Образовательная программа \\
«Прикладная математика»
\vspace{40mm}

\Large
\textbf{ОТЧЕТ} \\
\textbf{по лабораторной работе № 1} \\
\vspace{5mm}
\Large
По теме \\
\LARGE\textbf{«Теория погрешностей и машинная арифметика»}
\end{center}

\begin{center}
\vfill

\large
\begin{flushright}
\textbf{Выполнил} \\
студент группы БПМ211 \\
Кудряшов Максим Дмитриевич \\
\end{flushright}

\large
\begin{flushright}
\textbf{Проверил} \\
Брандышев Петр Евгеньевич \\
\end{flushright}

\large
\vspace{20mm}
Москва - 2024
\end{center}
\end{titlepage} % конец титульной страницы

\newpage
\tableofcontents
\newpage

\section{Расчет погрешности частичных сумм ряда (№ 1.1.8)}

\subsection{Формулировка}

1. Найти сумму ряда $S$ аналитически.

$$S = \sum\limits_{n=0}^\infty \frac{32}{n^2+9n+20}$$

2. Найти частичные суммы ряда $S_N$ при $N = 10, 10^2, 10^3, 10^4, 10^5$.

$$S_N = \sum\limits_{n=0}^N \frac{32}{n^2+9n+20}$$

3. Вычислить погрешности для каждого N.

4. Вычислить количество значащих цифр для каждого N.

5. Построить гистограмму зависимости верных цифр результата от $N$.

\subsection{Найдем сумму ряда аналитически}

Посчитаем значение предела в Wolfram:

\begin{figure}[h]
\frame{\includegraphics[width=0.3\linewidth]{../../calculate_sum/wolfram}}
\end{figure}

\subsection{Найдем частичные суммы ряда и их погрешности}

Для различных значений N вычислим частичные суммы, погрешности и количество значащих цифр.

\subsubsection{Код на Python}

\insertMyCode{python3}{../calculate_sum/calculate_sum.py}

\newpage
\subsection{Посчитанные данные}

\begin{minted}{python}
+--------+--------------------+------------------------+---------------------+
| N      | Значение           | Погрешность            | Число значащих цифр |
+--------+--------------------+------------------------+---------------------+
| 10     | 5.714285714285714  | 2.2857142857142856     | 0                   |
| 100    | 7.69230769230769   | 0.30769230769231015    | 1                   |
| 1000   | 7.968127490039844  | 0.03187250996015578    | 2                   |
| 10000  | 7.9968012794882295 | 0.003198720511770503   | 3                   |
| 100000 | 7.999680012799457  | 0.00031998720054282614 | 4                   |
+--------+--------------------+------------------------+---------------------+
\end{minted}

\begin{figure}[h]
    \includegraphics[width=0.5\linewidth]{../../calculate_sum/digits}
    \includegraphics[width=0.5\linewidth]{../../calculate_sum/errors}
\end{figure}

\section{Расчет погрешности матрицы (№ 1.9.2)}

\subsection{Формулировка}

Для матрицы $A$ решить вопрос о существовании обратной матрицы в следующих случаях:

1. элементы матрицы заданы точно;

2. элементы матрицы заданы приближенно с относительной погрешностью $\alpha = 0.05$ и $\beta = 0.1$

$$
A =
\begin{pmatrix}
    30 & 34 & 19 \\
    31.4 & 35.4 & 20 \\
    24 & 28 & 13 \\
\end{pmatrix}
$$

\subsection{Теория}

ДОПИСАТЬ ТЕОРИЮ, СКАЗАТЬ О ТЕОРЕМЕ

\subsection{Код на Python}

\insertMyCode{python3}{../matrix_det/matrix_det.py}

\subsection{Вывод программы}

\begin{minted}{text}
Определитель без погрешности 9.600000000000069

Минимальное значение определителя = -984.8728000000016
Максимальное значение определителя = 1027.8990000000008
При относительной погрешности 0.05 определитель может обратиться в 0

Минимальное значение определителя = -2965.2384
Максимальное значение определителя = 3032.2959999999985
При относительной погрешности 0.1 определитель может обратиться в 0
\end{minted}

\subsection{Выводы}

Если значения заданы точно, то определитель не равен 0, а следовательно существует обратная матрица.

Если лементы матрицы заданы приближенно с относительной погрешностью $\alpha = 0.05$ и $\beta = 0.1$, то определитель можно равняться нулю, значит обратная матрица можно не существовать.

\section{Нахождение машиннного нуля (№ 1.6, 1.7)}

\subsection{Формулировка}

Вычислить значения машинного нуля, машинной бесконечности, машинного эпсилон в режимах одинарной , двойной и расширенной точности на двух алгоритмических языках. Сравнить результаты.


\subsection{Код на Python}

\insertMyCode{python3}{../machine_values/machine_values.py}

\subsection{Код на C++}

\insertMyCode{c++}{../machine_values/machine_values.cpp}

\subsection{Вывод кода на Python}

\begin{minted}{text}
<class 'numpy.float32'> машинный ноль = 2^-150
<class 'numpy.float32'> машинная бесконечность = 2^128
<class 'numpy.float32'> машинное эпсилон = 2^-24

<class 'numpy.float64'> машинный ноль = 2^-1075
<class 'numpy.float64'> машинная бесконечность = 2^1024
<class 'numpy.float64'> машинное эпсилон = 2^-53

<class 'numpy.longdouble'> машинный ноль = 2^-1075
<class 'numpy.longdouble'> машинная бесконечность = 2^1024
<class 'numpy.longdouble'> машинное эпсилон = 2^-53
\end{minted}

\subsection{Вывод кода на C++}

\begin{minted}{text}
f машинный ноль = 2^-150
f машинная бесконечность = 2^128
f машинное эпсилон = 2^-24

d машинный ноль = 2^-1075
d машинная бесконечность = 2^1024
d машинное эпсилон = 2^-53

e машинный ноль = 2^-1075
e машинная бесконечность = 2^1024
e машинное эпсилон = 2^-53
\end{minted}

\subsection{Сравнение результатов}

Видно, что и машинный ноль, и машинное эпсилон, и машинная бесконечность совпадают для Python и для C++.




\end{document}
