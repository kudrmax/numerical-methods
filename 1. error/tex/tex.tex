\documentclass[a4paper,12pt]{article}
\usepackage[english, russian]{babel} % подключение языков
\usepackage[T2A]{fontenc}
\usepackage[utf8]{inputenc}
% \usepackage{minted}
\usepackage{verbatim}
\usepackage{graphicx}
\usepackage{caption}
\usepackage{subcaption}
\usepackage[left=2.5cm,right=2.5cm,top=2cm,bottom=2cm]{geometry}

\usepackage{geometry} % Меняем поля страницы
\geometry{left=1.5cm} % левое поле
\geometry{right=1.5cm} % правое поле
\geometry{top=1cm} % верхнее поле
\geometry{bottom=2cm} % нижнее поле

\newenvironment{longlisting}{\captionsetup{type=listing}}{}

\newenvironment{pseudolisting}
 {\begin{minipage}{\linewidth}\vspace*{\topsep}}
 {\vspace*{\topsep}\end{minipage}}

\begin{document}

\begin{titlepage}
  \begin{center}
    \large
     
    \textbf{Федеральное государственное автономное образовательное учреждение высшего образования}
    \vspace{0.5cm}
 
    НАЦИОНАЛЬНЫЙ ИССЛЕДОВАТЕЛЬСКИЙ УНИВЕРСИТЕТ \\ "ВЫСШАЯ ШКОЛА ЭКОНОМИКИ"
    \vspace{0.5cm}
     
    Московский институт электроники и математики имени А. Н. Тихонова 
     
    Программа "Прикладная математика"
    \vfill
     
     
    Кудряшов Максим Дмитриевич
    \vfill
 
    \textsc{Лаборатная работа}\\[5mm]
     
    {\LARGE Теория погрешностей и машинная арифметика\\[2mm]
    }
  \bigskip
     
    3 курс, группа БПМ203
\end{center}
\vfill
 

 
\hfill\begin{flushright}
  \textbf{Преподаватель:}\\
  Брандышев Петр Евгеньевич
\end{flushright}%
\vfill
 
\begin{center}
  Москва, 2021 г.
\end{center}
\end{titlepage}


\tableofcontents

\section{Расчет частичных сумм ряда}
\subsection{Формулировка задачи}
Дан ряд, надо найти сумму аналитически как предел частичных сумм, затем вычислить частичные суммы в зависимости от N и сравнить абсолютную погрешность и кол-во верных цифр в частичной сумме.
   $$S(N) = \sum_0^N a_n$$
   $$a_n = \frac{32}{n^2 +5n + 6}$$
   
\subsection{Аналитический расчет суммы ряда}
Выпишем формулу суммы ряда:
   $$S = \sum_{0}^{\infty} \frac{32}{n^2 +5n + 6}$$
Разделим знаменатель выражения на простые многочлены и на разные слагаемые:
   $$a_n =  \frac{32}{(n+2)(n+3)} = \frac{A}{n+2} + \frac{B}{n+3}$$
   $$$$
   $$a_n = \frac{A}{n+2} + \frac{B}{n+3} = \frac{An + 3A + Bn + 2B}{(n+2)(n+3)}$$
Вычислим значения в методе неопределенных коэффициентов:
   $$3A + 2B = 32, A=-B$$
   $$A=32, B=-32$$
Распишем первые несколько членов ряда и заметим, что соседние сокращаются:
   $$S(N) = \frac{32}{0+2} - \frac{32}{0+3} + \frac{32}{1+2} - \frac{32}{1+3}  ... + \frac{32}{n+2} - \frac{32}{n+3}$$
Перейдем к пределу и выведем аналитическое значение предела суммы:
   $$S(N) = \frac{32}{0+2} - \frac{32}{n+3}$$
   $$S = \lim_{s\rightarrow \infty}S_N = 16 - 0 = 16$$

\subsection{Код на Python}

\begin{longlisting}
% \inputminted{python}{src/series.py}
\end{longlisting}

\subsection{Результат работы программы}
\begin{longlisting}
\verbatiminput{output/series.txt}
\end{longlisting}

\subsection{Графики точности результата}
% \begin{figure}[H]
% \centering
% \begin{subfigure}{.5\textwidth}
%   \centering
%   % \includegraphics[width=\linewidth]{plots/series_error.png}
%   \caption{Абсолютная погрешность}
%   \label{fig:sub1}
% \end{subfigure}%
% \begin{subfigure}{.5\textwidth}
%   \centering
%   % \includegraphics[width=\linewidth]{plots/series_n_digits.png}
%   \caption{Кол-во значащих цифр}
%   \label{fig:sub2}
% \end{subfigure}
% \caption{Точность в зависимости от N}
% \label{fig:test}
% \end{figure}

\section{Квадратное уравнение}
\subsection{Формулировка задачи}
Дано квадратное уравнение. Предполагается, что один из коэффициентов уравнения (помечен $*$) получен в результате округления. Произвести теоретическую оценку погрешностей корней в зависимости от погрешности коэффициента. Вычислить корни уравнения при нескольких различных значениях коэффициента в пределах заданной точности, сравнить.
$$x^2+bx+c = 0$$
$$b = -30.9$$
$$c^* = 238.7 $$
    
\subsection{Теоретическая оценка погрешности}

Выпишем формулу уравнения и погрешности переменных и общую формулу для функций:
$$ax^2 + bx + c = 0$$
$$c^* = c \pm \Delta c$$
$$\bar{\Delta} x = |x| \bar{\delta} x$$
$$\bar{\Delta} f(x) = |f'(x)| \bar{\Delta} x$$
Оценим погрешность корня уравнения в зависимости от коэффициента $c$:
$$\frac{\bar{\Delta} f(x)}{|f(x)|} = \bar{\delta} f(x) = \frac{|xf'(x)|}{|f(x)|}\bar{\delta}x$$

$$\bar{\delta} x_{1,2} = |\frac{c}{x_{1,2}} \frac{\partial x_{1,2}}{\partial c} | \times \bar{\delta} c$$
Запишем общую формулу корня квадратного уравнения и вычислим производную:
$$x_{1,2} = \frac{-b \pm \sqrt{b^2-4ac}}{2a} $$

$$\frac{\partial x_{1,2}}{\partial c} = -\frac{1}{\sqrt{b^2-4ac}} = -10$$
Рассчитаем теоретические погрешности корней в зависимости от относительной погрешности $c$:
$$\bar{\delta} x_{1} = \frac{30.9}{15.5} \times 10 \times \bar{\delta} c = ~19.935 \times \bar{\delta} c $$
$$\bar{\delta} x_{2} = \frac{30.9}{15.4} \times 10 \times \bar{\delta} c = ~20.065 \times \bar{\delta} c$$
   
\subsection{Код на Python}
\begin{longlisting}
% \inputminted{python}{src/quadratic_eq.py}
\end{longlisting}

\subsection{Результат работы программы}
\begin{longlisting}
\verbatiminput{output/quadratic_eq.txt}
\end{longlisting}

\section{Машинная точность}
\subsection{Формулировка задачи}
Вычислить значения машинного нуля, машинной бесконечности и машинного эпсилон в режимах одинарной, двойной и расширенной точности на двух алгоритмических языках.
\subsection{Код на Python}

\begin{longlisting}
% \inputminted{python}{src/precision.py}
\end{longlisting}

\subsection{Результат работы программы}
\begin{longlisting}
\verbatiminput{output/precision.txt}
\end{longlisting}

\subsection{Код на C++}
\begin{longlisting}
% \inputminted{c++}{src/precision.cpp}
\end{longlisting}

\subsection{Результат работы программы}
\begin{longlisting}
\verbatiminput{output/precision_cpp.txt}
\end{longlisting}


\section{Вычисления с ограниченной разрядностью}
\subsection{Формулировка задачи}
Составить программу, моделирующую вычисления на ЭВМ с ограниченной разрядностью $m$.
Решить задачу о вычислении суммы ряда для случая $n=10000$, используя эту программу. Составить график зависимости погрешности от количества разрядов $m = \{4, 5, 6, 7, 8\}$.

\subsection{Код на Python}

\begin{longlisting}
% \inputminted{python}{src/series_fixed_precision.py}
\end{longlisting}

\subsection{Результат работы программы}
\begin{longlisting}
\verbatiminput{output/series_fixed_precision.txt}
\end{longlisting}

\subsection{Графики точности результата}
% \begin{figure}[H]
% \centering
% \begin{subfigure}{.5\textwidth}
%   \centering
%   % \includegraphics[width=\linewidth]{plots/series_fixed_error.png}
%   \caption{Абсолютная погрешность}
%   \label{fig:sub3}
% \end{subfigure}%
% \begin{subfigure}{.5\textwidth}
%   \centering
%   % \includegraphics[width=\linewidth]{plots/series_fixed_n_digits.png}
%   \caption{Кол-во значащих цифр}
%   \label{fig:sub4}
% \end{subfigure}
% \caption{Точность в зависимости от разрядности}
% \label{fig:test}
% \end{figure}

\end{document}